\chapter{Problem}
\label{chap:problem}

\section{Problem Statement}

The advence in new video technologies suggest new resolution size (
4k(3840$\times$2160px) or 8k(7680$\times$4320px)), an increasing in the frame
rate (60fps - 120 fps) and an increasing in the  color representation formats
(10 bits and 12 bits). Moreover, the advance in differents technologies, as
cloud computing, have promoted the use of video-based services, such as 
streaming, video-conference, video-on-demand, IPTV, broadcasting, among others,
which have generated an significant increment in the amount of information
neccesary for representing a video. By example, Table \ref{fig:youtube} shows
the bit-rate required for representing a live streaming video through YouTube
with a specific resolution\footnote{\url{https://support.google.com/youtube/answer/2853702}}.

\begin{table}[!h]
\centering
\begin{tabular}{|c|c|c|}
\hline
\textbf{Resolution} & \textbf{Bit-rate Min (Kbps)} & \textbf{bit-rate Max (Kbps)} \\
\hline
2560$\times$1440$@$60 & 9000 & 18000 \\
\hline
2560$\times$1440$@$30 &  6000 & 13000  \\
\hline
1920$\times$1080$@$60 & 4500 & 9000\\
\hline
1920$\times$1080 & 3000 & 6000 \\
\hline
\end{tabular}
\caption{Bit-rate requirements for some video resolutions in live
streaming video for YouTube}
\label{fig:youtube}
\end{table}

in addition to high bit-rate requeriments, there are limitations in bandwidth in
some geographic regions, for example, Latin America has a internet speed
connection of 7.26 Mbps on average for fixed networks and 4 Mbps for mobile
networks on average \cite{cepal}. The high bit-rates and bandwidth limitations
involve continuous efforts to maximize the coding efficiency, i.e. reducing the
bit rate, keeping the video quality.

The conventional video encoders, as HEVC or VP9, use differents techniques for
exploting the statistical, psychovisual and coding redundancies present in a
video. This techniques have been widely deployed and now is neccesary to adapt
to conventional video encoders others strategies that can to maximize the
compression capacity. 

The sparse representation emerge as alternative for improving the coding
eficiency, exploiting the sparsity propierties of signals for compressing with a
small amount of samples in terms of a base called dictionary. The sparse
representation is generated solving an optimization problem of $\ell_p$-norm of
sparse vector, such that lineal combination between the dictionary and the
sparse vector aproximate the original signal. A key problem of sparse
representation is the dictionary selection, because a good dictionary can to
guarantee a better signal reconstruction. The dictionary can be constructed with
samples extracted directly of the signal or can be trained for adapt it to
signal. The training process of dictionary consists of initialization step and
an iterative process of updating based on optimization problem of $\ell_p$-norm.
The updating process depends on convergence of the problem and should be
searched strategies for increasing the speed of convergence, due to the
dictionary training aggregate a significant computational cost.



\section{Problem Formulation}
How to optimize the training process of the dictionary of modeling of sparse
representation for video coding?

\section{Objectives}
\subsection{General Objective}
To Optimize the training process of dictionary for modelling the sparse
representation for video coding

\subsection{Specific objectives}
\begin{itemize}
\item Reducing the training time of a dictionary for sparse
representation based on video signals samples Reducir el tiempo de entrenamiento
\item Design a training model of dictionary that integrate perceptual coding
methods
\item To minimize the reconstruction error of the video signal when
using the traning model designed
\item To implement the designed model adapting to video coding
software 
\end{itemize}

